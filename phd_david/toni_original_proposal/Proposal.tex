\documentclass[12pt]{article}
%\documentclass[11pt]{article}

\usepackage{amsfonts,amssymb}
%\usepackage{amsmath}%,amscd,epsf,epsfig}
\usepackage[catalan]{babel}
\usepackage[latin1]{inputenc}
\usepackage{hyperref}
\usepackage{graphicx}
\usepackage{listings}

%%%%%%%%%%%%%%%%%%%%%%%% Structure definitions %%%%%%%%%%%%%%%%%%%%%%%%%%%

\renewcommand{\url}[1]{(\href{#1}{here})}

\newcommand{\bref}[1]{(\ref{#1})}

\newcommand{\eqn}[1]{(\ref{#1})}
\newcommand{\be}{\begin{equation}}
\newcommand{\ee}{\end{equation}}
\newcommand{\ben}{\begin{displaymath}}
\newcommand{\een}{\end{displaymath}}
\newcommand{\bea}{\begin{eqnarray}}
\newcommand{\eea}{\end{eqnarray}}
\newcommand{\bean}{\begin{eqnarray*}}
\newcommand{\eean}{\end{eqnarray*}}
\newcommand{\nn}{\nonumber \\}
\newcommand{\ba}{\begin{array}}
\newcommand{\ea}{\end{array}}
\newcommand{\bi}{\begin{itemize}}
\newcommand{\ei}{\end{itemize}}

%some of Rob's additions
\newcommand{\labell}[1]{\label{#1}\qquad_{#1}} %{\label{#1}}
\newcommand{\reef}[1]{(\ref{#1})}
\newcommand{\non}{\nonumber}
\newcommand{\pf}{\partial}
\newcommand{\bZ}{\bar{Z}}
\newcommand{\dZ}{\dot{Z}}
\newcommand{\dr}{\dot{r}}
\newcommand{\dbZ}{\dot{\bZ}}
\newcommand{\bz}{\bar{z}}
\newcommand{\dz}{\dot{z}}
\newcommand{\dbz}{\dot{\bz}}
\newcommand{\dU}{U^\dagger}
\def\cc{\mathbb{C}}
\newcommand{\df}{\textrm{d}}
%\newcommand{\str}{{\rm STr}}
%\newcommand{\tr}{{\rm Tr}}
\newcommand{\tr}{\mbox{Tr}}
\newcommand{\str}{\mbox{str}}

\newcommand{\mt}[1]{\textrm{\tiny #1}}
\def\lt {\lambda}
\def\rt {r}
\def\rhot {\rho}
\def \rvac{r_\mt{vac}}
\def\nc {N_\mt{c}}
\def\nf {N_\mt{f}}
\def\ua {U(1)_\mt{A}}
\def\tsix {T_\mt{D6}}
\def\ut {U_\mt{KK}}
\def\uh {U_\mt{T}}
\def\gym {g_\mt{YM}}
\newcommand{\nonsol}{D4-soliton}
\newcommand{\te}{t_\mt{E}}
\newcommand{\ct}{T_\mt{deconf}}
\newcommand{\tcc}{T_\mt{fund}}
\newcommand{\tb}{\bar{M}}
\newcommand{\msusy}{M_\mt{susy}}
\newcommand{\lm}{\l_\mt{match}}

\def\l{\lambda}
\def\a{\alpha}
\def\ap{\alpha'}
\def\b{\beta}
\def\g{\gamma}
\def\bg{\bar{\gamma}}
\def\G{\Gamma}
\def\d{\delta}
\def\s{\sigma}
\def\e{\epsilon}
\def\vt{\vartheta}
\def\vp{\varphi}
\def\T{\Theta}

%Introduced by Toni:

\def\otaula{\begin{tabular}}
\def\ctaula{\end{tabular}}


\renewcommand{\O}{\Omega}
\renewcommand{\L}{\Lambda}
\renewcommand{\t}{\theta}
%\renewcommand{\om}{\omega}



% Shortcuts added by Toni:

\def\bnum{\begin{enumerate}}
\def\enum{\end{enumerate}}
\newcommand{\num}[1]{\enub {#1} \enue}

\def\CR{\mathbb{R}}
\def\CM{\mathcal{M}}
\def\CC{\mathcal{C}}
\def\CE{\mathcal{E}}
\def\CQ{\mathcal{Q}}
\def\avall{\vskip 0.3cm}
\def\espai{\;\;\;\;\;\;\;}
\def\zespai{\;\;\;\;}
\def\s{\sigma}
\def\8M{$\CM_8$}
\def\k{\kappa}
\def\be{\begin{equation}}
\def\ee{\end{equation}}
\def\G{\Gamma}
\def\g{\gamma}
\def\ei{e^{\underline{i}}}
\def\ej{e^{\underline{j}}}
\def\e1{e^{\underline{1}}}
\def\1u{\underline{1}}
\def\2u{\underline{2}}
\def\iu{\underline{i}}
\def\0u{\underline{0}}
\def\e{\epsilon}
\def\target{$\CR^{1,1}\times \mathcal{M}_8$ }
\def\target2{$\CR^{1,1}\times \mathcal{M}_8$,}
\def\9G{\G_{\underline{9}}}
%\def\otaula{\begin{tabular}}
%\def\ctaula{\end{tabular}}
\def\yvec{\vec{y}}
\def\rvec{\vec{r}}
\def\xvec{\vec{x}}

\def\a{\alpha}
\def\b{\beta}
\def\undos{{1\over 2}}
\def\p{\partial}
\def\we{\wedge}

\def\xespai{\,, \zespai}

\def\tl{\tilde{L}}
\def\ads{$AdS_5 \times S^5$ }
\def\kap{$\kappa$-symmetry }
\def\tg{\tilde{\gamma}}

%%%%%%%%%%%%%%%%%% Calligraphic Letters %%%%%%%%%%%%%%%%%%%%%%%%%%%%%%%%%

\newcommand{\cala}{\mbox{${\cal A}$}}
\newcommand{\calb}{\mbox{${\cal B}$}}
\newcommand{\calc}{\mbox{${\cal C}$}}
\newcommand{\cald}{\mbox{${\cal D}$}}
\newcommand{\cale}{\mbox{${\cal E}$}}
\newcommand{\calf}{\mbox{${\cal F}$}}
\newcommand{\calg}{\mbox{${\cal G}$}}
\newcommand{\calh}{\mbox{${\cal H}$}}
\newcommand{\cali}{\mbox{${\cal I}$}}
\newcommand{\calj}{\mbox{${\cal J}$}}
\newcommand{\calk}{\mbox{${\cal K}$}}
\newcommand{\call}{\mbox{${\cal L}$}}
\newcommand{\calm}{\mbox{${\cal M}$}}
\newcommand{\caln}{\mbox{${\cal N}$}}
\newcommand{\calo}{\mbox{${\cal O}$}}
\newcommand{\calp}{\mbox{${\cal P}$}}
\newcommand{\calq}{\mbox{${\cal Q}$}}
\newcommand{\calr}{\mbox{${\cal R}$}}
\newcommand{\cals}{\mbox{${\cal S}$}}
\newcommand{\calt}{\mbox{${\cal T}$}}
\newcommand{\calu}{\mbox{${\cal U}$}}
\newcommand{\calv}{\mbox{${\cal V}$}}
\newcommand{\calw}{\mbox{${\cal W}$}}
\newcommand{\calx}{\mbox{${\cal X}$}}
\newcommand{\caly}{\mbox{${\cal Y}$}}
\newcommand{\calz}{\mbox{${\cal Z}$}}

\newcommand{\bfcalc}{\mbox{\boldmath ${\cal C}$}}


%%%%%%%%%%%%%%%%% Boldmath Letters %%%%%%%%%%%%%%%%%%%%%%%%%%%%%%%%%%%%%%

\newcommand{\bfe}{\mbox{\boldmath $E$}}
\newcommand{\bfb}{\mbox{\boldmath $B$}}
\newcommand{\bfpi}{\mbox{\boldmath $\Pi$}}
\newcommand{\bfp}{\mbox{\boldmath $P$}}
\newcommand{\bfna}{\mbox{\boldmath $\nabla$}}
\newcommand{\bfd}{\mbox{\boldmath $D$}}
\newcommand{\bfeh}{\mbox{\boldmath $\hat{E}$}}
\newcommand{\bfbh}{\mbox{\boldmath $\hat{B}$}}
\newcommand{\bfah}{\mbox{\boldmath $\hat{A}$}}
\newcommand{\bx}{\mbox{\boldmath $x$}}
\newcommand{\by}{\mbox{\boldmath $y$}}
\newcommand{\bX}{\mbox{\boldmath $X$}}
\newcommand{\bY}{\mbox{\boldmath $Y$}}
\newcommand{\bV}{\mbox{\boldmath $V$}}
\newcommand{\bU}{\mbox{\boldmath $U$}}

\newcommand{\bfK}{\mbox{\boldmath $K$}}
\newcommand{\bfz}{\mbox{\boldmath $\theta$}}
\newcommand{\bfZ}{\mbox{\boldmath $Z$}}
\newcommand{\bfr}{\mbox{\boldmath $\rho$}}
\newcommand{\bfxi}{\mbox{\boldmath $\xi$}}
\newcommand{\bfs}{\mbox{\boldmath $\sigma$}}
\newcommand{\bft}{\mbox{\boldmath $t$}}
\newcommand{\bfT}{\mbox{\boldmath $T$}}
\newcommand{\bfss}{\mbox{\boldmath $s$}}
\newcommand{\bfS}{\mbox{\boldmath $S$}}
\newcommand{\bfG}{\mbox{\boldmath $\Gamma$}}
\newcommand{\bfo}{\mbox{\boldmath $0$}}

\newcommand{\bbe}[1]{\mbox{${\mathbb E}^{#1}$}}
\newcommand{\bbr}[1]{\mbox{${\mathbb R}^{#1}$}}
%\newcommand{\bbz}[1]{\mbox{${\mathbb Z}^{#1}$}}
\newcommand{\bbz}[1]{\mbox{${\mathbb Z}_{#1}$}}
\newcommand{\bbi}[1]{\mbox{${\mathbb I}_{#1}$}}
\newcommand{\bbo}[1]{\mbox{${\mathbb O}_{#1}$}}

%%%%%%%%%%%%%%%%%%%%%% Miscellaneous  %%%%%%%%%%%%%%%%%%%%%%%%%%%%%%%%%%%

\newcommand{\nef}{{$\caln =4$} }

\newcommand{\dd}[3]{\mbox{$( #1 | \mbox{D} #2 \perp \mbox{D} #3)$}}
\newcommand{\ds}[3]{\mbox{$( #1 | D #2 \perp \calt #3)$}}
\newcommand{\para}{\parallel}
\newcommand{\inter}{\, \cap \,}
\newcommand{\su}[1]{$\calt #1\,$}

\newcommand{\adss}[2]{\mbox{$AdS_{#1}\times {S}^{#2}$}}
\newcommand{\rn}{Reissner-Nordstr\"{o}m }

\newcommand{\bra}[1]{\mbox{$\langle #1 |$}}
\newcommand{\ket}[1]{\mbox{$| #1 \rangle$}}
\newcommand{\braket}[2]{\mbox{$\langle #1  | #2 \rangle$}}
\newcommand{\proj}[1]{\ket{#1}\!\bra{#1}}

\newcommand{\Gu}[1]{\Gamma_{\underline{#1}}}
\newcommand{\Gn}{\Gamma_\natural}

\newcommand{\un}{\underline}
\newcommand{\pa}{\partial}
\newcommand{\fc}{\frac}
\newcommand{\trace}{\mbox{Tr}}
\newcommand{\sac}{\, , \qquad}
\newcommand{\eg}{{\it e.g. }}
\newcommand{\ie}{{\it i.e. }}
\newcommand{\sgn}[1]{\mbox{sgn}(#1)}
\newcommand{\com}[1]{[[[{\bf #1}]]]}
\newcommand{\ad}{\dot{a}}
\newcommand{\bd}{\dot{b}}
\newcommand{\ik}{{\it k}}
\newcommand{\ione}{{\it 1}}
\newcommand{\itwo}{{\it 2}}
\newcommand{\ifive}{{\it 5}}

\newcommand{\ph}{\hat{P}}
\newcommand{\xh}{\hat{X}}
\newcommand{\ve}{\varepsilon}


\newcommand{\ra}{\rightarrow}

\newcommand{\comment}[1]{{\bf [[[#1]]]}}

%\newcommand{\atmp}[3]{{\it Adv. Theor. Math. Phys.} {\bf #1} {(#2)} #3}
%\newcommand{\ijtp}[3]{{\it Int. J. Theor. Phys.} {\bf #1} {(#2)} #3}
%%%%%%%%%%%%%%%%%%%%%%%%%%% For This %%%%%%%%%%%%%%%%%%%%%%%
\def\pamu{\partial_{\mu}}
\def\spamu{\partial^{\mu}}
\def\spanu{\partial^{\nu}}
\def\panu{\partial_{\nu}}
\def\pasi{\partial_{\sigma}}
\def\parho{\partial_{\rho}}
\def\Udag{U^{\dagger}}
\def\vpi{\vec{\pi}}
\def\vsi{\vec{\sigma}}
\def\fpi{f_{\pi}}
\def\xvec{\vec{x}}
\def\hphi{\hat{\phi}}
%\newcommand{\ide}{\mbox{${\mathbb I}$}}
\newcommand{\ide}{\mbox{${\mathbb 1}$}}
\def\hr{\hat{r}}
\def\hphi{\hat{\phi}}
\def\hB{\hat{B}}
\def\w{\omega}
\def\hg{\hat{\gamma}}
\def\dt{\dot{\t}}
\def\xv{\vec{X}}
\def\yv{\vec{y}}
\def\vv{\vec{v}}
\def\uv{\vec{u}}
\def\wv{\vec{w}}
\def\pv{\vec{p}}
\def\bx{{\bar \xi}}
\def\bw{{\bar \w}}
\def\bvv{\vec{\bar v}}
\def\dx{\dot{x}}

\newcommand{\sla}[1]{\slash \!\!\!\! #1}
\def\hi{\hat{i}}
\def\hj{\hat{j}}
\def\bt{\bar{\theta}}

\def\ex{i \tau_2}

%\newcommand{\comment}[1]{{\bf [[[#1]]]}}


\newcommand{\tm}[1]{{\bf toni: #1}}
\newcommand{\ag}[1]{{\bf adan: #1}}
\def\snrd{SNR$_{deco}$}
\def\snrr{SNR$_{record}$}


\begin{document}


%%%%%%%%%%%%%%%%%%%%%%%%%%%%%%%%%%%%%%%%%%%%%%%%%%%%%%%%%%%%%%%%%%%%%%%%%%
%%%%%%%%%%%%%%%%%%%%%%%%%%%%%%%%%%%%%%%%%%%%%%%%%%%%%%%%%%%%%%%%%%%%%%%%%%
%%%%%%%%%%%%%%%%%%%%%%%%%%%%%%%%%%%%%%%%%%%%%%%%%%%%%%%%%%%%%%%%%%%%%%%%%%

%\title{PhD Projects: David Garca Garzn}
%\maketitle

\section{PROJECT I: Creation of HRTF database via simulation}

\subsection{Intro}

Present databases of HRTF present the following problems:
\begin{itemize}
\item They are calculated at a fixed distance between source and mics. These
leads to difficulty in determining the distance of the source in reproduction.

\item Present artifacts due to the errors in the measuring devices (loudspeaker
and microphones responses among them), which are difficult to equalize.

\item Lack of flexibility due to the use of specific heads or mannequins.

\item They are available for scientific use, but not industrial use.

\end{itemize}

Databases can be found at:

\begin{itemize}
\item \url{http://recherche.ircam.fr/equipes/salles/listen} It is the Listen Project,
funded by EU, and with AKG and Ircam as partners.
\item \url{http://interface.cipic.ucdavis.edu/index.htm} CIPIC / IDAV Database, UC Davis,
University of California
\item \url{http://www.itakura.nuee.nagoya-u.ac.jp/HRTF} Nagoya University, Japan,
\item \url{http://www.ais.riec.tohoku.ac.jp} Advanced Acoustic Information Systems Lab, Japan.
\end{itemize}

\subsection{Papers needed for state of the art}

\begin{itemize}
\item For a psychoacoustics book: Blauer, {\it Spatial Hearing}.

\item For a 2007 paper containing refs, \url{PROJECT-I/2007.pdf}

\item For a FDTD applied to HRTF, \url{PROJECT-I/FDTD !!!.pdf}

\item For one of the main papers on Binaural Technology by , there is
an old version by Moller, \url{PROJECT-I/Fundamentals of Binaural Tech.pdf}
and a newer one, \url{PROJECT-I/Updated Fundamentals of Binaural Tech.pdf},

\item For a study on the importance of head tracking, \url{PROJECT-I/Importance of Head motion.pdf}

\item For a good description, including refs to their papers, of how to simulate
the HRTF based on physical parametrization, the same website of CIPIC,
\url{http://interface.cipic.ucdavis.edu/index.htm/}

\end{itemize}


\subsection{Tentative Steps to follow}

\begin{enumerate}

\item Get to the state of the art.

\item Decide the method to implement, FDTD, FEM, BEM...

\item Implement it for a sphere, and test against the analytic well-known result.

\item If previous step is successful, look for fine-detail 3D models of heads.

\item Obtain Database. If computations are too long, perhaps use UPF cluster.

\item Analysis of results, extract and model relevant cues, use spherical harmonic
decomposition, etc.

\end{enumerate}


%%%%%%%%%%%%%%%%%%%%%%%%%%%%%%%%%%%%%%%%%%%%%%%%%%%%%%%%%%%%%%%%%%%%%%%%%%%%%%%%%%%%%%%
%%%%%%%%%%%%%%%%%%%%%%%%%%%%%%%%%%%%%%%%% PROJECT II %%%%%%%%%%%%%%%%%%%%%%%%%%%%%%%%%%
%%%%%%%%%%%%%%%%%%%%%%%%%%%%%%%%%%%%%%%%%%%%%%%%%%%%%%%%%%%%%%%%%%%%%%%%%%%%%%%%%%%%%%%

\newpage
\section{Project II: Ray-Tracing, Beam Tracing, others. Wave-field Synthesis?}

\subsection{Intro}

At the moment, we have only explored ray-tracing for rendering sound.
There are a few features that need to be tackled:

\begin{enumerate}

\item Implement diffraction, which is a purely wave phenomenon, but which can be
sometimes reproduced effectively by extending the geometrical methods.

\item Explore the possibility of tackling outdoors simulation. It is important because
we would then integrate with the Laboratori de Visualitzaci Urbana (Michele) of BM.

\item Examine possible alternative geometrical methods (Beam tracing...) to Ray-Tracing.

\item Examine if sound for Wave-Field Synthesis (WFS) exhibition can be generated
with RT or others. If not, implement it.

\item Integration with FDTD simulations. Produce a flexible Hybrid model.

\end{enumerate}


\subsection{Papers needed for state of the art}

\begin{itemize}

\item
For a very good pedagogical exposition of geometrical methods for sound rendering,
the course \url{PROJECT-II/Funkhouser-Course.pdf} and the paper
\url{PROJECT-II/Funkhouser-I.pdf} by Funkhouser.

\item For a book on ray-tracing, Glassner, {\it Introduction to Ray Tracing}.

\item For a thesis on WFS, by J. van der Vorm, \url{PROJECT-II/Thesis on WFS.pdf}.

\item For a comparison of WFS to Ambisonics, \url{PROJECT-II/WFS vs Ambisonics.pdf}.

\end{itemize}


\subsection{Tentative Steps to follow}

\begin{enumerate}

\item Get to the state of the art of geometrical methods, specially those
that are not ray-tracing, and making emphasis on their capability to
emulate diffraction.

\item Look for bibliography of outdoor simulation.

\item Decide if any of them performs better than RT, if so implement, if not
implement only diffraction to ray-tracing.

\item At some point, compare impulse responses to those measured by our Lab
(Giulio's project).

\item Explore the production of WFS compatible sound.

\item Explore the possibility of outdoor simulation.


\end{enumerate}

\section{Project III: Improve existing decoding algorithms}

\subsection{Intro}

By decoding, we understand the process of, given a set of 
pressure and velocities signals, 
At the moment, we are using a rather stardard decoding algorithm,
let us call it the {\it Venice algorithm}. See description in section
\ref{sec:venice_alg}.
It has some disadvantages:
\begin{itemize}
\item It does not take into account that the loudspeaker setup
might not be equiangular, as in 5.1. This makes the energy arriving
from the front much louder than from the back.
\item It reproduces sound in {\it all} speakers even if it is
a plane wave from infinity. This makes the sweetspot small
due to the precedence effect.
\item It is design to work well only for low frequencies, where phases
are measurable.
\end{itemize}

Alternatively, one can use the VBAP (or stereo panning) for
direct sound, which is great for localization purposes, as it
uses only the 2 speakers closer to the source (3 speakers in 3D).
However, it cannot be used for reverberant sounds where energy
arrives from all directions.

One way out of these two, is to obtain a compromise between 
2-speakers and all-speakers philosophies. See Gerzon papers
for it.

We will be particularly interested in a completely different philosophy.
It is based on determining how directional an audio stream is at a
given time interval, say by defining a parameter $\eta\in[0,1]$. 
Depending on $\eta$, we will decide the volume of all loudspeakers 
(low $\eta$ means non-directional, all speakers turned on; 
high $\eta$, only two speakers turned on).
The details of the volumes can be decided by us, which is 
a nice freedom to have...

In what follows, the 3 main algorithms just commented.
At the end, the bibliography.


\subsubsection{Algorithm 1: the {\it Venice algorithm}}
\label{sec:venice_alg}

This algorithm requires first of all the convolutions:
\be
p = i * g_p \sac v_x = i * g_x \sac v_y= i* g_y \,.
\ee
This can be done once and for all before executing the $\b$-dependent part
(even off-line).

{\bf NOTE: } If you don't want to perform convolutions, but rather use dry sound directly,
just define $p=v_x$ and $v_y=0$. This simulates a sound source behind you.
If you want it in front, just set $p=-v_x$, and $v_y=0$.

When we have the at our disposal $(p,v_x,v_y)$, we first perform a $\b$-rotation\footnote{With
this convention, $\beta>0$ simulates the sound moving counter-clockwise or, more precisely,
a listener rotating clock-wise.}
\bea
u_x &=& v_x\cos\b  - v_y\sin\b  \nn
u_y &=& v_x \sin\b  + v_y\cos\b \,,
\eea
and, finally, the signal at each of the channels is given by
\be
C_i =\undos \left[ p - ( u_x \cos\a_i + u_y \sin\a_i ) \right] \sac i=1,...N_L\,.
\ee

{\bf About volumes:} The way volumes are normalized at the moment is by computing
the complete signals $C_i$, finding the maximum value attained by all the $C_i$,
and then dividing all $C_i$ by a number such that the new maximum is $2^{15}$, so that
there's no clipping in a 16-bit wav output.

\subsubsection{Algorithm 2: Stereo Panning, a.k.a. VBAP-2D}

This is the simplest possible algorithm. It doesn't use the info about velocities, and it
can only spatialize direct sound, not reverberations.

The only input parameters are the angle $\b$ where sound is supposed
to come from, and the audio stream $p$. Then, it first determines
the two loudspeakers that leave $\b$ between them.
Name them speaker $a$ and $b$, located at angles $\a_a,\a_b$.
Arrange them such that
\be
\a_a < \b < \a_b \,.
\ee
Then, assign gains as follows
\be
g_1 = {\sin(\b-\a_a) \over \sin^2(\b-\a_a)+\sin^2(\a_b-\b) } \sac
g_2=  {\sin(\a_b-\b)  \over \sin^2(\b-\a_a)+\sin^2(\a_b-\b) }\,.
\ee
Then, in playback, mute all speakers except $C_a$ and $C_b$, which sound as follows
\be
C_a = g_2 p \sac C_b = g_1 p \,.
\ee


\subsubsection{Algorithm 3: directionality-based algorithm}

The inputs for this algorithm are (these are parameters that we must be able to tune
real-time, so that they can be adjusted):
\begin{enumerate}
\item
$n_b$: the number of samples within each bin.
\item
$interpol$: a parameter that can be 0 or 1, and determines whether interpolation is used or not.
\item
$\eta_{cutoff}$ is a cutoff for the directionality parameter.
\item
$\s$: controls the width of the Gaussian that governs the volume function.
\end{enumerate}
Default values are
\be
n_b=1000\sac interpol=1 \sac \eta_{cutoff}=0.92 \sac \sigma=10 \,.
\ee
The steps of the algorithm are
\begin{enumerate}
\item
Focus on the bin to which the present sample belongs. Determine the directionality coefficient of
this bin $\eta_1$, and the intensity vector of this bin $\vec{I}^{avg}_1$. This is simply done
by averaging over the bin:
\be
\vec{I}^{avg}_1 = {1 \over n_b} \sum_{i=1}^{n_b} {2 p_i \vec{v}_i \over p_i^2+\vec{v}_i^2 } \sac \eta_1=|\vec{I}_{avg}| \,.
%\sac \hat{d}_1 = {\vec{I}_{avg} \over |\vec{I}_{avg}|} \,.
\ee
Note that we always have $0\le \eta_1 \le 1$. Note that the sum only runs over samples for which
$p_i^2+\vec{v}_i^2 \neq 0$.
\item
If $interpol=1$, we need to compute the same two quantities, but in using the previous or, even beter, the following bin.
This produces $\eta_2$ and $\vec{I}^{avg}_2$.
\item
If $interpol=0$, then define
\be
\eta=\eta_1  \,.
\ee
\begin{itemize}
\item
If $\eta_1 \neq 0$, then
\be
\hat{d}={\vec{I}^{avg}_1 \over | \vec{I}^{avg}_1|} \,.
\ee
\item
If $\eta_1 = 0$, then $\hat{d}=\vec{0}$.
\end{itemize}


If $interpol=1$, then define
\be
\eta=l_2 \eta_1 + l_1 \eta_2 \,.
\ee
\begin{itemize}
\item I
f $\eta_1\neq 0$ or $\eta_2\neq 0$, then
\be
\hat{d}={ l_2 \vec{I}^{avg}_1 + l_1 \vec{I}^{avg}_2  \over |  l_2 \vec{I}^{avg}_1 + l_1 \vec{I}^{avg}_2  |   }\,.
\ee
where $l_{1,2}$ are the proportion of distances to the left/right extreme samples $i_{L,R}$ of the bins:
\be
l_1 = { i - i_L \over n_b} \sac l_2 ={i_R - i \over n_b} \,.
\ee
\item
If $\eta_1 = \eta_2= 0$, then $\hat{d}=\vec{0}$.
\end{itemize}

\item
If $\eta > \eta_{cutoff}$ then $\eta = \eta_{cutoff}$.

\item
Now we have the data needed to determine the gains of each loudspeaker. First, as in other algorithms,
define
\be
\hat{r}_a = (\cos \a_a,\sin \a_a) \sac a = 1,...,N_L \,.
\ee
Compute how close a speaker is to the direction of the incoming energy of that bin:
\be
x_a =- \hat{r}_a \cdot \hat{d} \sac a = 1,...,N_L \,.
\ee

\item
Determine the gains of each loudspeaker via the following formulae:
\bea
g^{temp}_a &=& f(x_a,\eta,\s) \sac a=1,...,N_L \nn
g_a &=& {g^{temp}_a \over \sum_{a=1}^{N_L} (g^{temp}_a)^2} \sac a=1,...,N_L \,.
\eea
where the volume control function is defined as
\be
f(x,\eta,\s) = \exp \left(   -  { (x-1)^2 \over \s} \, {\eta^2 \over (1-\eta)^2} \right) \,,
\ee
unless $\eta=0$, a case where we define $f(x,0,\s) = 1$.

\item
Finally, the signal at each loudspeaker is
\be
C_a = g_a p \sac a=1,...,N_L\,.
\ee

\end{enumerate}

\subsection{Papers needed for state of the art}

\begin{itemize}

\item For a good general introduction to B-format and decoding, Gerzon again: \url{PROJECT-III/Gerzon_summary.pdf}

\item For a mentioning of different decodings of B-format signals, Farina: \url{PROJECT-III/Farina.pdf}

\item The theoretical explanation of the meaning of the
parameter the measures how directional sound is, \url{PROJECT-III/Stanzial.pdf}.
The important section is sec. VI. About a paper that has tried to use 
this, in my opinion, incorrectly, \url{PROJECT-III/Pulkki.pdf}%{hola}

\item For a paper introducing the problem of low and high-frequency decoding
problems, \url{PROJECT-III/Gerzon_Surround_sound_psychoacoustics.pdf}.
More mathematically, \url{PROJECT-III/Gerzon.pdf}

\item For a great exposition of decoding methods, a PhD thesis
in French (could never find a copy in English...) at 
\url{PROJECT-III/Daniel_phd.pdf}

\end{itemize}

\section{Exercises}

\input{Exercises}


\end{document}
