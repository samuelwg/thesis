%
% No cal saber gaire LaTeX per escriure una tesi.
%
% Afegiu el vostre text a la plantilla, i deixeu que Knuth i
% Lamport s'encarreguin de l'aspecte.

% Per crear el document en .pdf només cal fer 'pdflatex
% aquestarxiu.tex', 'bibtex aquestarxiu' (sense extensió) i
% 'pdflatex aquestarxiu.tex' un altre cop. Hi ha moltes eines que
% permeten simplificar moltíssim la compilació en latex (v, p.ex.,
% rubber: http://www.pps.jussieu.fr/~beffara/soft/rubber/)
%

% [ Tots els caràcters que segueixen el signe %, fins a final de
%   línia, són comentaris i no es processen. ]
%
% En el bloc següent, delimitat per dues línies d'asteriscs,
% s'especifica el format del document. El podeu ignorar sense
% problemes si no sabeu LaTeX.
%

% *****************************************************************************

% Aquí s'especifica quin tipus de document voleu, i quines
% extensions (llengua, gràfics, fonts) fareu servir.

\RequirePackage{fix-cm}                 % Technicalities
\documentclass[10pt,b5paper,twoside,showtrims,openright]{memoir}
\usepackage[utf8]{inputenc} 		% codificació dels caràcters
\usepackage[english,catalan]{babel}	% localització de l'estil
\usepackage{url}			% macro \url per introduir adreces
\usepackage{amsmath}			% macros per matemàtiques
\usepackage{amssymb}			% macros per matemàtiques
\usepackage{amsfonts}			% macros per matemàtiques
\usepackage[pdftex]{graphicx}		% suport per figures
\usepackage{color}                      % per poder fer servir colors
\usepackage[colorlinks,bookmarks,pdftex,hyperfigures,breaklinks]{hyperref} % produeix enllaços clicables
\usepackage[T1]{fontenc}                % Nou esquema de codificació (recomanat)
\usepackage[round,authoryear]{natbib}  % per poder fer citacions Autor-any
%\usepackage{fixltx2e}                   % Technicalities

\DeclareGraphicsExtensions{.pdf,.png} % preference order

\hypersetup{
    pdfauthor={David García Garzón},
    pdftitle={PhD Thesis: TODO},
}
%% Format UPF, dues cares.
% Recordeu que les dimensions d'A4 i B5 són (210mm,297mm) i
% (176mm,250mm), respectivament.


\setulmarginsandblock{30mm}{*}{1.0}  % Marges verticals (3cm a dalt, 1.0*3cm a baix)
\setlrmarginsandblock{*}{30mm}{1.0} % Marges laterals (3cm a la dreta, 1.0*3cm l'esquerra)



% Per defecte la classe 'memoir' no numera les subseccions. Que les numeri:
\maxsecnumdepth{subsection}
\setsecnumdepth{subsection}
% Tampoc no inclou les subseccions a l'Índex. Que les inclogui:
\maxtocdepth{subsection}
\settocdepth{subsection}

% Estil dels capítols (consulteu estils disponibles a
% http://www.imf.au.dk/system/latex/artikler/MemoirChapStyles/MemoirChapStyles.pdf)
\chapterstyle{hangnum}

% Colors dels enllaços clickables (els que hi ha per defecte són molt
% kitsch)
\definecolor{LinkColor}{rgb}{0, 0, 0.3}
\definecolor{ExtLinkColor}{rgb}{0, 0.3, 0}
\hypersetup{citecolor=LinkColor,linkcolor=LinkColor,urlcolor=ExtLinkColor}


% **********************************************************************************

\input{toni_defs}

% 
% Aquí és on realment comença el document
%

\begin{document}

%% Llengua
%% ------------------------------------------------------------
%% Modifiqueu la següent comanda segons la llengua que utilitzeu
\selectlanguage{english}


%% Modifiqueu la següent comanda segons la llengua que utilitzeu
%% aquest document segueix el format suggerit per la UPF
%% vegeu http://www.upf.edu/bibtic/guiesiajudes/tesis/quart.html
\thispagestyle{empty}

\noindent
Llenceu aquesta pàgina i substituïu-la per aquella que us faciliti
la Unitat d'Informació i Projecció Institucionals (UIPI),
disponible al formulari següent\\
\url{http://www.upf.edu/uii/sgrafics/formulari_tesi.htm}
\cleartorecto

%% Portada
%% ------------------------------------------------------------
%% 
\begin{titlingpage}
\begin{center}
        \vspace*{1cm} % Ad libitum
  {\Huge Como hacer una tesis y no jubilarte en el intento}

        \vspace{8mm} % Ad libitum
        {\LARGE David García Garzón}

        \vspace*{\fill} % Omple l'espai vertical tant com es pugui
        {\Large Tesi Doctoral UPF / 2012}

        \vspace*{\fill}
        {\normalsize Supervisada per}\\[2mm]
        {\Large Dr Toni Mateos}\\[1mm]
        {\Large Dr Vicente Lòpez}\\[1mm]
        {\large Departament de Tecnologia}

        \vspace{2cm}
%        {\normalsize Altres informacions que calgui incloure aquí}
\end{center}
\newpage

%% Espai reservat per DL i ISBN
%% ------------------------------------------------------------
\noindent
\raisebox{0.2ex}{\textcopyright}\  2012
David García Garzón.\\
Dipòsit Legal:\\
ISBN:

\vspace{1cm}
\footnotesize

\noindent
\includegraphics{cc-by-sa}

\noindent
This thesis is a free document.
You are able to distribute it under the terms of the license
Creative Commons - Attribution - Share Alike

\noindent
Aquesta tesi és un document lliure.
El podeu distribuir segons els termes de la llicència
Creative Commons - Atribució - Compartir en iguals condicions

\end{titlingpage}
\frontmatter

\cleartorecto % Comença a la pàgina dreta següent


%% Dedicatòria (opcional)
%% ------------------------------------------------------------
\thispagestyle{empty}
\vspace*{\stretch{1}}
\begin{flushright}
  \LARGE
  \itshape
Dedicatòria
\end{flushright}
\vspace*{\stretch{4}}

\cleartorecto

%% Acknowledgments
%% ------------------------------------------------------------
\chapter*{Acknowledgments}
\ldots

\cleartorecto
\cleartorecto
\thispagestyle{chapter}
\section*{Abstract}
\selectlanguage{english}
This is the abstract of the thesis in English.  Please, use less
than 150 words.

\selectlanguage{catalan}
\vspace*{\fill}
\section*{Resum}
Vet aquí el resum de la tesi en català.  Si us plau, utilitzeu
menys de 150 paraules.
\vspace*{\fill}
\selectlanguage{english}

\cleartorecto

\pagestyle{headings}
%% Prefaci o pròleg
%% ---------------------------------------------------------------------- 
\chapter{Preface}
Introducció a la tesi que acostuma a ressenyar-ne els mèrits, el
valor, o també situar-la dins un context i unes circumstàncies
determinades.

\vspace{1cm}
Aquest document es pot descarregar de
\begin{quotation}
%\url{http://upf.edu/dtip/~autor/tesi.pdf}
\url{http://upf.edu/dtip/~autor/tesi.pdf}
\end{quotation}
\cleartorecto


\tableofcontents % presenta la taula de continguts
%\clearpage      % presenta la taula de figures
%\listoffigures
%\clearpage
%\listoftables   % presenta la taula de taules


%% Cos de la tesi
%% ---------------------------------------------------------------------
\mainmatter


\input{content}


\backmatter

\bibliographystyle{upfstyle} % 
\bibliography{../BinauralSynthesis} % el meu arxiu .bib de referències (sense
                        % extensió)
%% bibliografia
% En aquest cas, l'escrivim nosaltres a mà. En general és molt més
% convenient fer servir BibTeX.
%\begin{thebibliography}{tesi}
%  \bibitem[Gar66]{GARDNER66}
%    Martin Gardner.
%    \newblock \emph{More Mathematical Puzzles and Diversions}.
%    \newblock Penguin Books
%    \newblock (ISBN 0--14--020748--1), 1966.
%
%  \bibitem[GMS94]{GOOSSENS94}
%    Michel Goossens, Frank Mittelbach and Alexander Samarin.
%    \newblock \emph{The LaTeX Companion}.
%    \newblock Addison-Wesley Publishing Company
%    \newblock (ISBN 0--201--54199--8), 1994.
%
%  \bibitem[FAR85]{fary85}
%    José Luis Cantero Rada, \emph{El Fary}.
%    \newblock \emph{Yo detesto al hombre blandengue}
%    \newblock Entrevista a Televisió Espanyola,
%    octubre 1985.
%  \bibitem[ONS44]{onsager1944} Lars Onsager, \newblock
%    \emph{Crystal Statistics. I. A Two-Dimensional Model with an Order-Disorder Transition},
%    \newblock Physical Review \textbf{65}(3-4),
%    p 118--149, 1944.
%\end{thebibliography}

\clearpage

%% índex de noms
%\pagestyle{index}
%\printindex
%\cleardoublepage

\pagestyle{empty}
\null\vfil

%\begin{adjustwidth}{1in}{1in}
\begin{center}
{\Large Colophon}
\end{center}
\begin{center}
	Aquesta tesi s'ha escrit en \LaTeX\ fent servir l'editor
        \texttt{vim}.\\
	El cos de la lletra és XXX i a s'han utilitzat les fonts
	YYY.
\end{center}
%\end{adjustwidth}
\vfil

\end{document}
% vim:set tw=66:
